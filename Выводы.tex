\section*{Заключение}
В работе были освоены методы проведения квантово-химических рассчетов в программной среде GAMESS и получены навыки чтения лог-файлов:
\begin{enumerate}
    \item установлено, что при проведении оптимизации выбор системы координат влияет на скорость сходимости;
    \item на основании электронного распределение сделаны выводы об ароматичности молекулы;
    \item из предложенных различных комплексов молекулы с водой определены стабильные, а также рассчитаны энергии образующихся водородных связей; 
    \item с помощью модели поляризуемого континуума определена константа диссоциации. Полученное значение близко к экспериментальному;
    \item найдено переходное состояние и определена возможность его существования при комнатной температуре; 
    \item определены энергии и интенсивности электронных переходов методами TD и CIS. При попытке сопоставить результаты возникла неоднозначность, поэтому выбор метода рассчета будет зависеть от рассматриваемой системы и требуемой точности;
    \item рассчитаны колебательные частоты, определены валентные, плоские и неплоские колебания. Полученные частоты валентных колебаний для CH-связи близки к экспериментальным значениям.
\end{enumerate}
\addcontentsline{toc}{section}{Заключение}