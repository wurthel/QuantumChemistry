\subsection{Теоретическая информация}
\subsubsection{TDDFT-метод}
Метод DFT непригоден для исследования характеристик возбужденных состояний. Поэтому используют другой подход: систему рассматривают в присутствии зависящих от времени потенциалов (электрические и магнитные поля). В TDDFT влияние таких полей на молекулу может быть изучено для получения таких характеристик, как энергии возбуждения, частотно-зависимые характеристики отклика и спектры фотопоглощения. 