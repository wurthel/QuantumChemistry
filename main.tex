 % XeLaTeX document
\documentclass[12pt,a4paper]{article}

% Редактируем: конфигурация, личные настройки: имя, название предмета и пр. для титульной страницы и метаданных документа здесь
\newcommand{\university}{Санкт-Петербургский политехнический университет Петра Великого}
\newcommand{\faculty}{Институт физики, нанотехнологий и телекоммуникаций}
\newcommand{\department}{Высшая инженерно-физическая школа}
\newcommand{\city}{Санкт-Петербург}
\newcommand{\num}{}
\newcommand{\docname}{Отчет по лабораторной работе №8, вариант 16}
\newcommand{\subject}{Определение энергий и интенсивностей электронных переходов молекул хиназолина методом TDDFT}
\newcommand{\tutorname}{И. М. Соколов}
\newcommand{\studentname}{В. Х. Салманов}
\newcommand{\group}{3430302/60201}

% Не редактируем: используемые пакеты
\include{settings/packages}

% Не редактируем: параметры используемых пакетов и не только
\include{settings/preferences}

% водяной знак для обозначения статуса документа
%\newwatermark[allpages,color=red!40,angle=45,scale=3,xpos=0,ypos=0]{DRAFT}

\begin{document}
% Не редактируем: Титульная страница (формируется автоматически из заданной конфигурации)
\include{templates/titlepage}

% Не редактируем: Страница содержания (формируется автоматически из subsection, paragraph и пр., указанных в content.tex)
% Содержание
\setcounter{tocdepth}{1} % Show sections
\setcounter{tocdepth}{2} % + subsubsections
\setcounter{tocdepth}{3} % + paragraphs
\setcounter{tocdepth}{4} % + paragraphs
\setcounter{tocdepth}{5} % + subparagraphs
\tableofcontents
\newpage



% Редактируем: всё остальное: вступление, др. этапы, заключение, приложение
\section*{Цели и задачи}
\paragraph{Цель работы:} освоить методы квантово-химических рассчетов в программной среде GAMESS.

\paragraph{Задачи:}
\begin{enumerate}
    \item провести расчет в одной точке и оптимизацию геометрии методами HF и DFT;
    \item сравнить результаты оптимизации в z-матричном и картезианском представлениях;
    \item проанализировать электронное разделение в $\pi$-сопряженной системе;
    \item определить структуры комплексов молекулы с водой и рассчитать энергии образующихся водородных связей;
    \item определить константу диссоциации молекулы с использованием модели поляризуемого континуума;
    \item определить геометрию переходного состояния для заданного переходного процесса;
    \item определить энергии и интенсивности электронных переходов методами TD и CIS и сравнить результаты;
    \item рассчитать спектр колебательных частот молекулы.
\end{enumerate}
\addcontentsline{toc}{section}{Цели и задачи}

\newpage
\section{Проведение SP-расчета и оптимизация геометрии методом HF молекулы изопропанола}
\subimport{lab1/}{main.tex}

\newpage
\section{Проведение оптимизация геометрии методом DFT молекулы этиленгликоля}
\subimport{lab2/}{main.tex}

\newpage
\section{Анализ электронного разделения в молекуле орто-фенантролина}
\subimport{lab3/}{main.tex}

\newpage
\section{Оптимизация геометрии молекулы этиленгликоля в Z-матричном представлении}
\subimport{lab4/}{main.tex}

\newpage
\section{Определение структуры комплексов пропанола с молекулой воды и энергии образующейся водородной связи}
\subimport{lab5/}{main.tex}

\newpage
\section{Определение константы диссоциации молекулы нитроуксусной кислоты}
\subimport{lab6/}{main.tex}

\newpage
\section{Поиск седловой точки молекулы трифторметанола в барьере поворота вокруг C-O-связи}
\subimport{lab7/}{main.tex}

\newpage
\section{Определение энергий и интенсивностей электронных переходов молекул хиназолина методом TDDFT}
\subimport{lab8/}{main.tex}

\newpage
\section{Определение энергий и интенсивностей электронных переходов молекул хиназолина методом HF/6-31G-CIS}
\subimport{lab9/}{main.tex}

\newpage
\section{Определение колебательных частот молекулы 1,4-дифторбензол}
\subimport{lab10/}{main.tex}

\newpage
\section{Выводы}
Метод конфигурационного взаимодействия является альтернативой TD-методам для расчета энергии возбуждения перехода. При попытке сопоставить переходы, рассчитанные различными методами, могут возникнуть неоднозначности: энергия перехода могут быть примерно равно, но тип возбужденного состояния и конфигурационный состав -- нет. 

% Не редактируем: Страница библиографии (формируется автоматически из книжек, указанных в refs.bib и пометок \cite{имя_источника} в тексте)
\newpage
\printbibliography[title=Перечень использованных источников]
\addcontentsline{toc}{subsection}{Перечень использованных источников}
\end{document}
