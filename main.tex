 % XeLaTeX document
\documentclass[12pt,a4paper]{article}

% Редактируем: конфигурация, личные настройки: имя, название предмета и пр. для титульной страницы и метаданных документа здесь
\newcommand{\university}{Санкт-Петербургский политехнический университет Петра Великого}
\newcommand{\faculty}{Институт физики, нанотехнологий и телекоммуникаций}
\newcommand{\department}{Высшая инженерно-физическая школа}
\newcommand{\city}{Санкт-Петербург}
\newcommand{\num}{}
\newcommand{\docname}{Отчет по лабораторной работе №1}
\newcommand{\subject}{Молекулярное моделирование}
\newcommand{\tutorname}{И. М. Соколов}
\newcommand{\studentname}{В. Х. Салманов}
\newcommand{\group}{3430302/60201}

% Не редактируем: используемые пакеты
% настройка кодировки, шрифтов и русского языка
\usepackage{fontspec}
\usepackage{polyglossia}

% рабочие ссылки в документе
\usepackage{hyperref}

% графика
\usepackage{graphicx}
\usepackage{tikz}

% поворот страницы
\usepackage{pdflscape}

% качественные листинги кода
\usepackage{minted}
\usepackage{listings}
\usepackage{lstfiracode}

% отключение копирования номеров строк из листинга, работает не во всех просмотрщиках (в Adobe Reader работает)
\usepackage{accsupp}
\newcommand\emptyaccsupp[1]{\BeginAccSupp{ActualText={}}#1\EndAccSupp{}}
\let\theHFancyVerbLine\theFancyVerbLine
\def\theFancyVerbLine{\rmfamily\tiny\emptyaccsupp{\arabic{FancyVerbLine}}}

% библиография
\bibliographystyle{templates/gost-numeric.bbx}
\usepackage{csquotes}
\usepackage[parentracker=true,backend=biber,hyperref=false,bibencoding=utf8,style=numeric-comp,language=english,autolang=other,citestyle=gost-numeric,defernumbers=true,bibstyle=gost-numeric,sorting=ntvy,]{biblatex}

% установка полей
\usepackage{geometry}

% нумерация картинок по секциям
\usepackage{chngcntr} 

% дополнительные команды для таблиц
\usepackage{booktabs}

% для заголовков
\usepackage{caption} 
\usepackage{titlesec}
\usepackage[dotinlabels]{titletoc}

% разное для математики
\usepackage{amsmath, amsfonts, amssymb, amsthm, mathtools}
\usepackage{braket}


% водяной знак на документе, см. main.tex
\usepackage[printwatermark]{xwatermark} 

% использование различных спец. символов
\usepackage{gensymb}

% таблицы
\usepackage{tabularx}
\newcolumntype{Y}{>{\centering\arraybackslash}X}

% гиперссылки
\usepackage{hyperref}

% межстрочный интервал
\sloppy
\linespread{1.2}
\usepackage{multirow}
\usepackage{graphicx}

\usepackage{import}

% Не редактируем: параметры используемых пакетов и не только
% настройки polyglossia
\setdefaultlanguage{russian}
\setotherlanguage{english}

% локализация
\addto\captionsrussian{
  \renewcommand{\figurename}{Рисунок}%
  \renewcommand{\partname}{Глава}
  \renewcommand{\contentsname}{\centerline{Содержание}}
  \renewcommand{\listingscaption}{Листинг}
}

% основной шрифт документа
\setmainfont{CMU Serif}

% перечень использованных источников
\addbibresource{refs.bib}

% настройка полей
\geometry{top=2cm}
\geometry{bottom=2cm}
\geometry{left=3cm}
\geometry{right=1.5cm}
\geometry{bindingoffset=0cm}

% настройка ссылок и метаданных документа
\hypersetup{unicode=true,colorlinks=true,linkcolor=red,citecolor=green,filecolor=magenta,urlcolor=cyan,
    pdftitle={\docname},   	    
    pdfauthor={\studentname},      
    pdfsubject={\subject},      		        
    pdfcreator={\studentname}, 	       
    pdfproducer={Overleaf}, 		     
    pdfkeywords={\subject}
}

% настройка подсветки кода и окружения для листингов
\usemintedstyle{colorful}
\newenvironment{code}{\captionsetup{type=listing}}{}

% шрифт для листингов с лигатурами
\setmonofont{FiraCode-Regular.otf}[
    Path = templates/,
    Contextuals=Alternate
]

% оформления подписи рисунка
\captionsetup[figure]{labelsep = period}

% подпись таблицы
\DeclareCaptionFormat{hfillstart}{\hfill#1#2#3\par}
\captionsetup[table]{format=hfillstart,labelsep=newline,justification=centering,skip=-10pt,textfont=bf}

% путь к каталогу с рисунками
\graphicspath{{fig/}}

% Внесение titlepage в учёт счётчика страниц
\makeatletter
\renewenvironment{titlepage} {
 \thispagestyle{empty}
}
\makeatother

\counterwithin{figure}{subsection}
\counterwithin{table}{subsection}

\titlelabel{\thetitle.\quad}

% для удобного конспектирования математики
\mathtoolsset{showonlyrefs=true}
\theoremstyle{plain}
\newtheorem{theorem}{Теорема}[subsection]
\newtheorem{proposition}[theorem]{Утверждение}
\theoremstyle{definition}
\newtheorem{corollary}{Следствие}[theorem]
\newtheorem{problem}{Задача}[subsection]
\theoremstyle{remark}
\newtheorem*{nonum}{Решение}

% настоящее матожидание
\newcommand{\MExpect}{\mathsf{M}}

% объявили оператор!
\DeclareMathOperator{\sgn}{\mathop{sgn}}

% перенос знаков в формулах (по Львовскому)
\newcommand*{\hm}[1]{#1\nobreak\discretionary{} {\hbox{$\mathsurround=0pt #1$}}{}} 

% формулы
\newcommand{\mequation}[1]{
\begin{equation}
\begin{split}
\begin{gathered}
#1
\end{gathered}
\end{split}
\end{equation}
}

% водяной знак для обозначения статуса документа
%\newwatermark[allpages,color=red!40,angle=45,scale=3,xpos=0,ypos=0]{DRAFT}

\begin{document}
% Не редактируем: Титульная страница (формируется автоматически из заданной конфигурации)
\begin{titlepage}	% начало титульной страницы

	\begin{center}		% выравнивание по центру

		\large \university \\
		\large \faculty \\
		\large \department \\[5cm]
		% название института, затем отступ 6см
		
		\LARGE \textbf \subjecttype \\ % тип работы (например: курсовая работа)
		\LARGE \subject \\[0.5cm] % название работы, затем отступ 0,5см
		\large \docname \num \\[4.1cm]
		%\large Тема работы\\[5cm]

	\end{center}


	\begin{flushright} % выравнивание по правому краю
		\begin{minipage}{0.25\textwidth} % врезка в половину ширины текста
			\begin{flushleft} % выровнять её содержимое по левому краю

				\large\textbf{Работу выполнил:}\\
				\large \studentname \\
				\large {Группа:} \group \\
				
				\large \textbf{Преподаватель:}\\
				\large \tutorname

			\end{flushleft}
		\end{minipage}
	\end{flushright}
	
	\vfill % заполнить всё доступное ниже пространство

	\begin{center}
	\large \city \\
	\large \the\year % вывести дату
	\end{center} % закончить выравнивание по центру

\end{titlepage} % конец титульной страницы

\vfill % заполнить всё доступное ниже пространство


% Не редактируем: Страница содержания (формируется автоматически из subsection, paragraph и пр., указанных в content.tex)
% Содержание
\setcounter{tocdepth}{1} % Show sections
\setcounter{tocdepth}{2} % + subsubsections
\setcounter{tocdepth}{3} % + paragraphs
\setcounter{tocdepth}{4} % + paragraphs
\setcounter{tocdepth}{5} % + subparagraphs
\tableofcontents
\newpage



% Редактируем: всё остальное: вступление, др. этапы, заключение, приложение
\section*{Цели и задачи}
\paragraph{Цель работы:} освоить методы квантово-химических рассчетов в программной среде GAMESS.

\paragraph{Задачи:}
\begin{enumerate}
    \item провести расчет в одной точке и оптимизацию геометрии методами HF и DFT;
    \item сравнить результаты оптимизации в z-матричном и картезианском представлениях;
    \item проанализировать электронное разделение в $\pi$-сопряженной системе;
    \item определить структуры комплексов молекулы с водой и рассчитать энергии образующихся водородных связей;
    \item определить константу диссоциации молекулы с использованием модели поляризуемого континуума;
    \item определить геометрию переходного состояния для заданного переходного процесса;
    \item определить энергии и интенсивности электронных переходов методами TD и CIS и сравнить результаты;
    \item рассчитать спектр колебательных частот молекулы.
\end{enumerate}
\addcontentsline{toc}{section}{Цели и задачи}

\newpage
\section{Проведение SP-расчета и оптимизация геометрии методом HF молекулы изопропанола}
\subimport{lab1/}{main.tex}

\newpage
\section{Проведение оптимизация геометрии методом DFT молекулы этиленгликоля}
\subimport{lab2/}{main.tex}

\newpage
\section{Анализ электронного разделения в молекуле орто-фенантролина}
\subimport{lab3/}{main.tex}

\newpage
\section{Оптимизация геометрии молекулы этиленгликоля в Z-матричном представлении}
\subimport{lab4/}{main.tex}

\newpage
\section{Определение структуры комплексов пропанола с молекулой воды и энергии образующейся водородной связи}
\subimport{lab5/}{main.tex}

\newpage
\section{Определение константы диссоциации молекулы нитроуксусной кислоты}
\subimport{lab6/}{main.tex}

\newpage
\section{Поиск седловой точки молекулы трифторметанола в барьере поворота вокруг C-O-связи}
\subimport{lab7/}{main.tex}

\newpage
\section{Определение энергий и интенсивностей электронных переходов молекул хиназолина методом TDDFT}
\subimport{lab8/}{main.tex}

\newpage
\section{Определение энергий и интенсивностей электронных переходов молекул хиназолина методом HF/6-31G-CIS}
\subimport{lab9/}{main.tex}

\newpage
\section{Определение колебательных частот молекулы 1,4-дифторбензол}
\subimport{lab10/}{main.tex}

\newpage
\section{Выводы}
В данном соединении преобладают $n-\pi$-переходы. Соединение поглощает в диапазоне длин волн от 200 до 400 нм. Максимум поглощения приходится на 247 нм.

% Не редактируем: Страница библиографии (формируется автоматически из книжек, указанных в refs.bib и пометок \cite{имя_источника} в тексте)
\newpage
\printbibliography[title=Перечень использованных источников]
\addcontentsline{toc}{subsection}{Перечень использованных источников}
\end{document}
