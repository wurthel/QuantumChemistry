\section{Теоретическая информация}
\subsection{Модель поляризуемого континуума (PCM)}
\ldots

\subsection{Расчет $pK_a$}
Константа диссоциации кислоты ($K_a$) — константа равновесия реакции диссоциации кислоты на катион водорода и анион кислотного остатка.Чаще вместо самой константы диссоциации $K_a$ используют величину $pK_a$, которая определяется как отрицательный десятичный логарифм самой константы $K_a$:
\mequation{
    pK_a= -\lg \left(K_{a}\right)
}

Величина $pK_a$ связана с разностью энергий между депротонированной и нейтральной формами $D = E^{-} - E^{0}$ следующим теоретическим соотношением (при $T = 293$):
\mequation{
    \left(pK_a\right)_{T}^{'} = \frac{1}{2.3RT}\left(D - \frac{5}{2}RT - 0.41345\right)
}
где 0.41345 Хатри – энергия сольватации протона в водной среде. 

Для получения реального приближенного значения $pK_a$ по полученному значению   используются эмпирические соотношения, зависящие от конкретного варианта метода расчета. В данном случае такое эмпирическое соотношение имеет следующий вид:
\mequation{
    \left(pK_a\right)_{T} = -6.9 + 0.55\left(pK_a\right)_{T}^{'}
}
 	
