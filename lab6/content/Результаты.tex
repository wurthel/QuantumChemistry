\section{Результаты}
Для нейтральной и депротонированной форм молекулы была проведена оптимизация методом B3LYP/6-31G. В полученных минимумах были проведены SP-вычисления методом B3LYP/6-311G++(2d,p).

\begin{table}[H]
    \caption{Значения энергии для форм молекулы (в Хартри)} 
    \label{tab:my-table}
        \begin{center}
            \begin{tabular}{|c|c|c|}
            \hline
            Форма соединения & \begin{tabular}[c]{@{}c@{}}Полная энергия\\ с учетом сольватации\end{tabular} & Энергия сольватации \\ \hline
            Нейтральная & -433.538121 & -0.019305 \\ \hline
            Депротонированная & -433.090787 & -0.096514 \\ \hline
            \end{tabular}
        \end{center}
\end{table}

Вычислим значение $pK_a$.
\mequation{
    D = E^{-} - E^{0} = 0.447334 \\
    \left(pK_a\right)_{T} = 1.23
}

Экспериментальное значение $pK_a = 1.68$\footnote{\url{https://pubchem.ncbi.nlm.nih.gov/bioassay/448096\#sid=103702226&section=Version}}