\section{Теоретическая информация}
\subsection{Нормальные координаты}
Система нормальных координат является удобной системой координат, в которой выражения для кинетической и потенциальной энергии имеет очень простой вид:
\mequation{
    T = \frac{1}{2}\sum\limits_{i=1}^{3N}\dot{q_i}^2 \\
    U = \frac{1}{2}\sum\limits_{i=1}^{3N}\lambda_i q_i^2
} 
где $q_i$ -- координата ($i \bmod 3$) атома в системе нормальных координат, N -- число атомов, $\lambda_i$ -- собственные числа гессиана. 

Чтобы рассчитать колебательные частоты, можно придерживаться следующей схемы, реализованной в программе $Gaussian$\footnote{К сожалению найти, как считаются собственные частоты в программе $GAMESS$, мне не удалось.}.

Сперва гессиан переводится в масс-взвешенные картезианские координаты
\mequation{
    H_{MWC ij} = \frac{H_{CART ij}}{\sqrt{m_i m_j}}, i, j = 1, 2, \ldots, 3N
}
После этого гессиан диагонализируется
\mequation{
    \boldsymbol{U}^{\dagger}\boldsymbol{HU} = \boldsymbol{\Lambda} = \left( \begin{array}{cccc}
        \lambda_1 & 0 & \ldots & 0\\
        0 & \lambda_2 & \ldots & 0\\
        0 & 0 & \ldots & \lambda_{3N}\\
        \end{array}
        \right)
}
где \textbf{U} -- матрица перехода в нормальные координаты.

Колебательные частоты связаны с собственными числами гессиана следующим образом:
\mequation{
    v_i = \sqrt{\frac{\lambda_i}{4\pi^2}}, i = 1, 2, \ldots, 3N
}